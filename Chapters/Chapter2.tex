% Chapter 2

\chapter{Stato dell'arte} % Write in your own chapter title
\label{Capitolo 2}
\lhead{Capitolo 2. \emph{Stato dell'arte}} % Write in your own chapter title to set the page header

\par
In questo capitolo sar\`{a} descritto lo stato dell'arte riguardanti gli studi e le analisi effettuate relative ad informazioni di tipo geolocalizzato riguardanti gli esseri umani; questa tipologia di dati abbraccia diversi ambiti, tra i quali l'ambito sociale e l'ambito scientifico.
\par
L'interesse che gli accademici che si occupano di sociologia e comportamento umano \`{e} dovuto principalmente al modo in cui gli spostamenti delle persone condizionino le amicizie e relazioni tra loro, ma soprattutto come l'analisi dei dati sia spesso contrastante con l'impressione che hanno le persone stesse dei loro spostamenti, come esposto da Nathan Eagle ed altri in \citep{Reference1}.
\par
Un'altro ambito in cui lo studio di informazioni di tipo geolocalizzato \`{e} quello scientifico, dove si pu\`{o} trovare l'applicazione di diverse discipline matematiche per effettuare l'analisi di dati grezzi; tra queste la teoria dei grafi e la statistica sono sicuramente le maggiormente utilizzate. Gli studiosi che si occupano di scienze sono interessati all'aggregazione dei dati grezzi in dati pi� strutturati, per poter, ad esempio, essere in grado di predirre sia le posizioni delle persone nel futuro, sia le relazioni di amicizia tra gli utenti nel tempo.
\\
Infine, utilizzando i dati strutturati in maniera adeguata, \`{e} possibile creare un sistema di raccomandazione e stimare la similitude tra utenti basandosi solamente su informazioni di tipo geolocalizzato.

\section{Ambito sociale}
\section{Ambito scientifico}
\subsection{Predizione delle traiettorie}
\subsection{Predizione delle connessioni tra utenti}
\subsection{Sistema di raccomandazione}
\subsection{Similitude tra utenti}